%%%%%%%%%%%%%%%%%%%%%%%%%%%%%%%%%%%%%%%%%
% Masters/Doctoral Thesis 
% LaTeX Template
% Version 2.5 (27/8/17)
%
% This template was downloaded from:
% http://www.LaTeXTemplates.com
%
% Version 2.x major modifications by:
% Vel (vel@latextemplates.com)
%
% This template is based on a template by:
% Steve Gunn (http://users.ecs.soton.ac.uk/srg/softwaretools/document/templates/)
% Sunil Patel (http://www.sunilpatel.co.uk/thesis-template/)
%
% Template license:
% CC BY-NC-SA 3.0 (http://creativecommons.org/licenses/by-nc-sa/3.0/)
%
%%%%%%%%%%%%%%%%%%%%%%%%%%%%%%%%%%%%%%%%%

%----------------------------------------------------------------------------------------
%	PACKAGES AND OTHER DOCUMENT CONFIGURATIONS
%----------------------------------------------------------------------------------------

\documentclass[
12pt, % The default document font size, options: 10pt, 11pt, 12pt
oneside, % Two side (alternating margins) for binding by default, uncomment to switch to one side
english, % ngerman for German
onehalfspacing, % Single line spacing, alternatives: onehalfspacing or doublespacing
%draft, % Uncomment to enable draft mode (no pictures, no links, overfull hboxes indicated)
%nolistspacing, % If the document is onehalfspacing or doublespacing, uncomment this to set spacing in lists to single
%liststotoc, % Uncomment to add the list of figures/tables/etc to the table of contents
%toctotoc, % Uncomment to add the main table of contents to the table of contents
%parskip, % Uncomment to add space between paragraphs
%nohyperref, % Uncomment to not load the hyperref package
headsepline, % Uncomment to get a line under the header
%chapterinoneline, % Uncomment to place the chapter title next to the number on one line
%consistentlayout, % Uncomment to change the layout of the declaration, abstract and acknowledgements pages to match the default layout
]{MastersDoctoralThesis} % The class file specifying the document structure
\usepackage{minted}
\usepackage{multicol}
\usepackage{graphicx}
\usepackage{float}
\usepackage[utf8]{inputenc} % Required for inputting international characters
\usepackage[T1]{fontenc} % Output font encoding for international =

\usepackage{booktabs}
\usepackage{multirow}
\usepackage[final]{pdfpages}
\usepackage{mathpazo} % Use the Palatino font by default
\usepackage{changepage}
\usepackage{caption}
\usepackage[list=false]{subcaption}
\usepackage[backend=biber,style=apa,natbib=true]{biblatex} % Use the bibtex backend with the authoryear citation style (which resembles APA)

\addbibresource{honsDoc.bib} % The filename of the bibliography

\usepackage[autostyle=true]{csquotes} % Required to generate language-dependent quotes in the bibliography

%----------------------------------------------------------------------------------------
%	MARGIN SETTINGS
%----------------------------------------------------------------------------------------

\geometry{
	paper=a4paper, % Change to letterpaper for US letter
	inner=2.5cm, % Inner margin
	outer=2cm, % Outer margin
	%bindingoffset=0.3cm, % Binding offset
	top=2cm, % Top margin
	bottom=2.5cm, % Bottom margin
	%showframe, % Uncomment to show how the type block is set on the page
}

%----------------------------------------------------------------------------------------
%	THESIS INFORMATION
%----------------------------------------------------------------------------------------

\thesistitle{The Qualities of Games for Use in Education} % Your thesis title, this is used in the title and abstract, print it elsewhere with \ttitle
\supervisor{Prof. Günther \textsc{Drevin}} % Your supervisor's name, this is used in the title page, print it elsewhere with \supname
\examiner{} % Your examiner's name, this is not currently used anywhere in the template, print it elsewhere with \examname
\degree{Bachelor of Science Honours in Computer Science and Information Technology} % Your degree name, this is used in the title page and abstract, print it elsewhere with \degreename
\author{Joshua \textsc{Esterhuizen}} % Your name, this is used in the title page and abstract, print it elsewhere with \authorname
\addresses{} % Your address, this is not currently used anywhere in the template, print it elsewhere with \addressname

\subject{Computer Sciences} % Your subject area, this is not currently used anywhere in the template, print it elsewhere with \subjectname
\keywords{} % Keywords for your thesis, this is not currently used anywhere in the template, print it elsewhere with \keywordnames
\university{\href{http://http://www.nwu.ac.za/}{North-West University}} % Your university's name and URL, this is used in the title page and abstract, print it elsewhere with \univname
\department{\href{http://http://natural-sciences.nwu.ac.za/computer-sciences-and-information-systems}{School of Computer Science and Information Systems}} % Your department's name and URL, this is used in the title page and abstract, print it elsewhere with \deptname
%\group{\href{}{}} % Your research group's name and URL, this is used in the title page, print it elsewhere with \groupname
\faculty{\href{http://natural-sciences.nwu.ac.za/}{Faculty of Natural and Agricultural Sciences}} % Your faculty's name and URL, this is used in the title page and abstract, print it elsewhere with \facname




\newcommand{\titleAfrikaans}{Die Eienskappe van Speletjies vir Gebruik in Onderwys}

\AtBeginDocument{
\hypersetup{pdftitle=\ttitle} % Set the PDF's title to your title
\hypersetup{pdfauthor=\authorname} % Set the PDF's author to your name
\hypersetup{pdfkeywords=\keywordnames} % Set the PDF's keywords to your keywords
}

\begin{document}

\frontmatter % Use roman page numbering style (i, ii, iii, iv...) for the pre-content pages

\pagestyle{plain} % Default to the plain heading style until the thesis style is called for the body content

%----------------------------------------------------------------------------------------
%	TITLE PAGE
%----------------------------------------------------------------------------------------

\begin{titlepage}
\begin{center}
\includegraphics[scale=0.4]{Figures/nwu.png}\\[.4cm]

{\scshape\LARGE \univname\par}\vspace{0.5cm} % University name
\textsc{\Large School of Computer Science and Information Systems}\\[0.5cm] % Thesis type

\HRule \\[0.4cm] % Horizontal line
{\huge \bfseries \ttitle\par}\vspace{0.4cm} % Thesis title
\HRule \\[1.0cm] % Horizontal line
 
\begin{minipage}[t]{0.4\textwidth}
\begin{flushleft} \large
\emph{Author:}\\
\href{mailto:joshua.esterhuizen27@gmail.com}{\authorname} \\ % Author name - remove the \href bracket to remove the link
\textit{30285976}
\end{flushleft}
\end{minipage}
\begin{minipage}[t]{0.4\textwidth}
\begin{flushright} \large
\emph{Supervisor:} \\
\href{mailto:Gunther.Drevin@nwu.ac.za}{\supname} % Supervisor name - remove the \href bracket to remove the link  
\end{flushright}
\end{minipage}\\[3cm]
 
\vfill

\large \textit{A thesis submitted in fulfilment of the requirements for the degree of \\ \degreename}\\[0.3cm] % University requirement text
\large \textit{for the}\\
\large \textit{ITRI671 module}\\
\vfill

{\large \today}\\[4cm] % Date
%\includegraphics{Logo} % University/department logo - uncomment to place it
 
\vfill
\end{center}
\end{titlepage}

%----------------------------------------------------------------------------------------
%	ABSTRACT PAGE
%----------------------------------------------------------------------------------------

\begin{abstract}
\addchaptertocentry{\abstractname} % Add the abstract to the table of contents
Education is a system that had its' last major revision during the industrial age which saw instructional institutions act like factories with the goal of propelling students through the system as quick as possible. Due to this, a curriculum is often standardised and used as a means to teach all students in the same way. The issue with this is that some students may not learn effectively with the modes of education presented. One possible implementation to aid in solving this problem is the inclusion of digital and computer-based serious games into the modes of instruction.
\\\\
This study was focused on identifying possible qualities these types of games should possess to be used effectively within an educational environment. In order to reach this goal, a literature study was conducted on the fields of pedagogy, ludology and gamification with each playing a role in better understanding the means of developing these qualities. Additionally, several case studies concerning serious game artefacts were examined in which the design philosophies of the serious games were discussed and examined to identify any correlation between instructional theories. The results of this section of the project are several key qualities required as well as a brief discussion on how different topics could be approached.
\\\\
Furthermore, this project included the development of an artefact in the form of a digital computer game. These aspects of the project were conducted simultaneously with minimal correlation. The artefact was developed using the Unity Editor and C\# as the main developmental tools. The result of this is a short computer game that has a focus on simple puzzle-solving, platforming and action-based gameplay.
\\\\
\textbf{Keywords:} Game Design and Development, Gamification, Ludology, Pedagogy, Serious Games, Unity 


\end{abstract}

\begin{extraAbstract}
\addchaptertocentry{\abstractnameExtra} % Add the abstract to the table of contents
Onderwys is 'n stelsel wat sy laaste groot hersiening tydens die industriële era gehad het, waarin onderriginstellings soos fabrieke opgetree het met die doel om studente so vinnig as moontlik deur die stelsel te dryf. As gevolg hiervan word 'n kurrikulum dikwels gestandaardiseer en gebruik as 'n manier om alle studente op dieselfde manier te onderrig. Die probleem hiermee is dat sommige studente dalk nie effektief leer met die onderrigmetodes wat aangebied word nie. Een moontlike implementering om te help om hierdie probleem op te los, is die insluiting van digitale en rekenaargebaseerde ernstige speletjies in die onderrigmodusse.
\\\\
Hierdie studie was gefokus op die identifisering van moontlike eienskappe wat hierdie tipe speletjies behoort te besit om effektief binne 'n opvoedkundige omgewing gebruik te word. Ten einde hierdie doel te bereik, is 'n literatuurstudie oor die velde van pedagogie, ludologie en gamification gedoen, met elkeen wat 'n rol gespeel het om die maniere om hierdie eienskappe te ontwikkel beter te verstaan. Verder is verskeie gevallestudies oor ernstige speletjie-artefakte ondersoek waarin die ontwerpfilosofieë van die ernstige speletjies bespreek en ondersoek is om enige korrelasie tussen onderrigteorieë te identifiseer. Die resultate van hierdie afdeling van die projek is verskeie sleuteleienskappe wat vereis word, asook 'n kort bespreking oor hoe verskillende onderwerpe benader kan word.
\\\\
Verder het hierdie projek die ontwikkeling van 'n artefak in die vorm van 'n digitale rekenaarspeletjie ingesluit. Hierdie aspekte van die projek is gelyktydig uitgevoer met minimale korrelasie. Die artefak is ontwikkel met behulp van die Unity Editor en C\# as die belangrikste ontwikkelingsinstrumente. Die resultaat hiervan is 'n kort rekenaarspeletjie wat 'n fokus het op eenvoudige legkaartoplossing, platforms en aksie-gebaseerde spel.
\\\\
\textbf{Keywords:} Speletjieontwerp en -ontwikkeling, Gamification, Ludologie, Pedagogie, Ernstige Spele, Unity 


\end{extraAbstract}



%--------------------------------------------------------------------------------
%	LIST OF CONTENTS/FIGURES/TABLES PAGES
%----------------------------------------------------------------------------------------

\tableofcontents % Prints the main table of contents

\listoffigures % Prints the list of figures

\listoftables % Prints the list of tables

%----------------------------------------------------------------------------------------
%	ABBREVIATIONS
%----------------------------------------------------------------------------------------
%
%\begin{abbreviations}{ll} % Include a list of abbreviations (a table of two columns)
%
%\textbf{LAH} & \textbf{L}ist \textbf{A}bbreviations \textbf{H}ere\\
%\textbf{WSF} & \textbf{W}hat (it) \textbf{S}tands \textbf{F}or\\
%
%\end{abbreviations}
%
%----------------------------------------------------------------------------------------
%	THESIS CONTENT - CHAPTERS
%----------------------------------------------------------------------------------------

\mainmatter % Begin numeric (1,2,3...) page numbering

\pagestyle{thesis} % Return the page headers back to the "thesis" style

% Include the chapters of the thesis as separate files from the Chapters folder
% Uncomment the lines as you write the chapters

% Chapter 1

\chapter{Introduction} % Main chapter title

\label{Chapter1} % For referencing the chapter elsewhere, use \ref{Chapter1} 

%----------------------------------------------------------------------------------------

% Define some commands to keep the formatting separated from the content 
\newcommand{\keyword}[1]{\textbf{#1}}
\newcommand{\tabhead}[1]{\textbf{#1}}
\newcommand{\code}[1]{\texttt{#1}}
\newcommand{\file}[1]{\texttt{\bfseries#1}}
\newcommand{\option}[1]{\texttt{\itshape#1}}

%----------------------------------------------------------------------------------------

\section{Project Description}

\section{Project Contents}
\subsection{Problem Description and Background}
\subsection{Overview of Related Literature}
\subsection{Research Question and Expected Outcomes}

\section{Aims and Objectives of Project}

\section{Procedures and Methods of Investigation}
\subsection{Research Paradigm and Methodology Used}
\subsection{Collection of Data}
\subsection{Development of Artefact}
\subsection{Integration with Other Projects}

\section{Approach to Project Management and Project Plan}
\subsection{Provisional Project Plan}

\begin{figure}[p]
\centering
\includegraphics[scale=1.3]{gantt}
\caption{Provisional Project Plan}
\end{figure}

\subsection{Scope}
\subsection{Contributions of Each Group Member}

\section{Development Platform, Resources and Environments}
\subsection{Development Platform}
\subsection{Resources}
\subsection{Environments}

\section{Ethical Implications of Project}

\section{Provisional Chapter Division}


% Chapter 2

\chapter{Literature Review} % Main chapter title

\label{Chapter2} % For referencing the chapter elsewhere, use \ref{Chapter1} 


\section{Introduction}

\section{An Issue with Instructional Education}

\section{Pedagogy and Learning Theories}
\subsection{Learner's at the Core of Learning}
\subsection{Merrill's First Principles of Instruction}
\subsection{The Influence of Motivation}


\section{An Approach Through Ludology}
\subsection{What is Ludology}
\subsection{Serious Games}
\subsection{Games as Simulation}

\section{Gamification}

\section{Existing Gamified Teaching Systems and Educational Games}

\section{Potential Effects of Game-Based Learning}

\section{Conclusion and Summary}



% Chapter 3

\chapter{Development of Artefact and Article} % Main chapter title

\label{Chapter3} % For referencing the chapter elsewhere, use \ref{Chapter1} 

This chapter describes the artefact that was developed in a group effort as well as the process followed for the writing of the academic article.

\section{Description of Artefact}
This research project was split into two main aspects: 
\begin{enumerate}
\item An individual research study into a chosen topic \textit{(this being the qualities needed for a computer game to be used in education)}
\item A group effort to develop a digital computer game.
\end{enumerate}

\noindent For the first aspect of this project, the resulting outcome was an academic article titled, \textit{"Linking Gamification, Ludology and Pedagogy: How to Use Serious Games for Various Knowledge Domains"}. As stipulated in the title, the fields of gamification, ludology and pedagogy were the focus of this research aspect of the project. In this article, the question posed by this project is again reiterated and sequentially answered. The article is attached to this document under Appendix  \ref{AppendixC}. 
\\\\
The artefact accompanying this research project is a digital computer came which was developed as part of a group effort as mentioned previously. While this artefact is separate from the individual research studies conducted by its' group members, it may have been influenced by the topics being researched. The artefact is titled \textit{"Puzzle Ball: Spherical Shadows"} and is further categorised as a 3D puzzle platformer with aspects of action gameplay played in first person. A playable version of the artefact is available on the website itch.io website\footnote{\url{https://josh-scg.itch.io/puzzle-ball-spherical-shadows}} and the source code of the artefact is available on GitHub\footnote{\url{https://github.com/GCWehmeyer/Spherical_Shadows}}$^{,}$\footnote{\url{https://github.com/Josh-SCG/Spherical_Shadows}}

\section{Artefact Life Cycle Followed and its Different Phases}
% Briefly provide general background on the life cycle that was followed and why it is appropriate to this artefact.

\section{Description of the Development of the Artefact}
%Comprehensive report on the way in which each phase of the life cycle was applied in the development of this artefact.




% Chapter 4

\chapter{Results} % Main chapter title

\label{Chapter4}  

\section{Artefact}
The artefact that was developed is a computer game and is titled \textit{"Puzzle Ball: Spherical Shadows"} and is a self-contained application that can be launched standalone without any additional software needed. While it is fairly lightweight on hardware resources for a computer game, it does require more resources than a typical application and may stutter on older systems due to the effects used. In terms of the genres it encompass, it would full under the platformer, puzzle and action genres for games and as such is described as a 3D puzzle platformer with aspects of action gameplay played in first person.
\\\\
Once the application is launched, the user is taken to the \textit{Main Menu} screen (depicted below) where clicking the text \textit{Play} triggers a function to start the first level - that being the the Tutorial. Each group member was responsible for the design of an individual level where the Tutorial/First level being the level undertaken in this project. The reason for this is that the research aspect of this project aligns with the goals of a tutorial level. 

\begin{figure}[H]
\centering
\begin{subfigure}{0.5\textwidth}
  \centering
  \includegraphics[width=1\linewidth]{Figures/menu.png}
  \caption{The Main Menu of the Artefact}
\end{subfigure}%
\begin{subfigure}{0.5\textwidth}
  \centering
  \includegraphics[width=1\linewidth]{Figures/start.png}
  \caption{The Starting Point of the Level}
\end{subfigure}
\caption{Starting the Game}
\end{figure}

\noindent One the tutorial level has started, the user is shown the basic character movement controls on the right side of the screen and can continue forward. Text based signs were used throughout the level to explain various mechanics or tasks to be completed - an example of this is shown in Figure 4.2 below.

\begin{figure}[H]
\centering
\centerline{\includegraphics[scale=0.33]{tutSignEg.png}}
\caption{A Further Example of the Tutorial Signs}
\end{figure}

\noindent The user must then make there way to a central area, referred to as the Hub Room, where they are given the choice in what order they would like to complete the five main tutorials. These will be discussed and shown below. As the user completes the several tutorials, a blockade that restricted movement to the next level is slowly removed as shown below.

\begin{figure}[H]
\centering
\begin{subfigure}{0.3\textwidth}
  \centering
  \includegraphics[width=1\linewidth]{Figures/barrier5.png}
  \caption{No Tutorials Completed}
\end{subfigure}%
\begin{subfigure}{0.3\textwidth}
  \centering
  \includegraphics[width=1\linewidth]{Figures/barrier3.png}
  \caption{Two Tutorials Completed}
\end{subfigure}
\begin{subfigure}{0.3\textwidth}
  \centering
  \includegraphics[width=1\linewidth]{Figures/barrier0.png}
  \caption{All Tutorials Completed}
\end{subfigure}
\caption{Progress Blocker in Tutorial}
\end{figure}

\noindent Once all the tutorials in this level are completed and the blockade removed, the user is allowed to move towards the final room where they must throw the ball object into the centre of the desk-like object with a particle effect on it.

\begin{figure}[H]
\centering
\centerline{\includegraphics[scale=0.33]{Figures/endgoal.png}}
\caption{The End Goal Object}
\end{figure}

\noindent Once the ball object collides with the end goal object, the user is taken to the next level where the main aim to to traverse the level and throw the ball object into the next end goal. This process is followed for two additional levels for three in total  after which the user is taken to a scene where the group members and their contributions are listed with a prompt to close the application.

\begin{figure}[H]
\centering
\centerline{\includegraphics[scale=0.33]{Figures/credits.png}}
\caption{The Credits Screen}
\end{figure}

\section{Article}
A literary analysis on the fields of pedagogy, ludology and gamification was conducted and an article written in order to answer the question posed; \textit{What qualities and principles can be applied to a video game to allow it to be used in a learning environment as a means to provide better engagement among certain students by providing an enjoyable delivery of information?}
\\\\
This question was then refined and phrased as \textit{What qualities are required for a serious game to be effectively used in an educational environment on various topics?} within the article.
\\\\
The results of the research aspect of this project is the answer to these questions which were synthesised from various case studies and the literature analysis conducted.

\section{Conclusion} 
% Chapter 5

\chapter{Reflection} % Main chapter title

\label{Chapter5} % For referencing the chapter elsewhere, use \ref{Chapter1} 

%What did you learn while completing this project?  Refer to the decisions that you took, and evaluate them.  Analyse the strong and weak points of the product and the process that you followed.  What would you do differently?


%Did you achieve the objectives that you set for the project?  Discuss.


%How successful were you in managing the project and meeting target dates?  Discuss.


%Comments on group work.  Responsibilities and contribution of each group member.  Problems with cooperation and contributions, and dealing with these problems.


%The reflection must be structured logically.


%Conclude the reflection with a summary that is a critical evaluation of the process followed with the honours project


   

%----------------------------------------------------------------------------------------
%	BIBLIOGRAPHY
%----------------------------------------------------------------------------------------

\printbibliography[heading=bibintoc,title=Reference List]


%----------------------------------------------------------------------------------------
%	THESIS CONTENT - APPENDICES
%----------------------------------------------------------------------------------------

\appendix % Cue to tell LaTeX that the following "chapters" are Appendices

% Include the appendices of the thesis as separate files from the Appendices folder
% Uncomment the lines as you write the Appendices

% Appendix A

\chapter{Ethics Form} % Main appendix title

\label{AppendixA} % For referencing this appendix elsewhere, use \ref{AppendixA}
%\includepdf[pages=-,pagecommand={},scale=0.3]{Ethics.pdf}
\begin{minipage}{\textwidth}
\includepdf[scale=0.85,pages=1,offset=0 -3cm]{ethics.pdf}
\end{minipage}
\includepdf[pages=2- ,pagecommand={}, offset=0 -0cm]{ethics.pdf}
% Appendix B

\chapter{Research Proposal} % Main appendix title

\label{AppendixB} % For referencing this appendix elsewhere, use \ref{AppendixA}
%\includepdf[pages=-,pagecommand={},scale=0.3]{proposal.pdf}
\begin{minipage}{\textwidth}
\includepdf[scale=0.85,pages=1,offset=0 -3cm]{proposal.pdf}
\end{minipage}
\includepdf[pages=2- ,pagecommand={}, offset=0 -0cm]{proposal.pdf}
% Appendix C

\chapter{Research Project Article} % Main appendix title

\label{AppendixC} % For referencing this appendix elsewhere, use \ref{AppendixC}
%\includepdf[pages=-,pagecommand={},scale=0.3]{article.pdf}

\begin{minipage}{\textwidth}
\includepdf[scale=0.9,pages=1,offset=0 -4cm]{article.pdf}
\end{minipage}
\includepdf[pages=2- ,pagecommand={}, offset=0 -0cm]{article.pdf}


%----------------------------------------------------------------------------------------

\end{document}  
