% Chapter 5

\chapter{Reflection} % Main chapter title

\label{Chapter5} % For referencing the chapter elsewhere, use \ref{Chapter5} 
This chapter will focus on the personal reflections held about this project at its' completion. This process is conducted by answering several questions about the project.

\section{What was Learnt Throughout this Project}
In terms of what was learnt during the duration of this project, there were several new technologies and software in addition to a vast amount of knowledge from various fields.
\\\\
On the software side of things, the largest one was the Unity Editor, specifically the 2019.4.22f build of it, which was the main tool in developing the artefact. The choice of using Unity over another game engine, such as Unreal or Godot, was mainly due to the programming language it used (that being C\#). This was the language that most of the group, excluding myself, was familiar with and most experienced with when compared to the alternatives. Following this, C\# was also a new language learnt for this project and while the initial usage of it was difficult, with the experience  Java, another object-oriented language, it became easy to write and implement the scripts needed. 
\\\\
Another technology made use of, and consequently learnt, is GitHub. This was used for version control and was the simplest way to handle it. The GitHub Desktop application was used for facilitating this. 
\\\\
\LaTeX was another technology used in the latter half the project. It is a software tool that allows for the creation of PDF's and was used for the Article as well as the final documentation. While it was initially hard to pick up and use, it also became easier with time. It allowed for these respective documents to be presented very neatly and should have been used earlier on in the project's development. 
\\
A fair amount of information concerning game design was also learnt form how psychology plays a factor in the choices game designers take when developing a game or user interface. While a fair amount of this knowledge was used when designing the artefact, there is still much more that could have been incorporated - such as the aforementioned use of psychology to style the game better.
\\\\
Additionally, various knowledge in the fields used for the research article was learnt. These were fields concerning educational and instructional theories, the study of games as well as turning tasks into some form of game be it digital or not. These topics were interesting to examine and read about and were a tremendous help on how to approach any future educational endeavours such as tutoring. 
\\\\
As this project was described to be under \textit{Entertainment Computing}, being able to analyse other computer games was vital to garner a proper understanding of how to go about the development of both the artefact and article. This process was heavily aided by Prof. Günther Drevin where we were tasked with various presentations on the field of ludology itself and on the analysis of a game - Myst.

\section{Development of Artefact}
The primary aim of this study was stated as, \textit{to identify what qualities and principles can be applied to a video game to allow it to be used in a learning environment as a means to provide better engagement among certain students by providing an enjoyable delivery of information.}

The objectives needed for the article development were given as:
\begin{enumerate}
\item A literature study will need to be performed to gather information with a focus on:
\begin{enumerate}
\item Ludology, narratology and simulation to better understand the academia centred around this project;
\item Other implementations of serious games and the qualities they possess;
\item The impact and effects of games in early development as this is the main “target audience” if this project were to be implemented as well as in a more general sense;
\item Previous attempts to integrate game use in learning.
\end{enumerate}
\item Collect examples of games that employ some form of teaching 
(\textit{Where objective 1-b focuses on examples already discussed in an academic sense, this objective will make use of more informal analysis})
\end{enumerate}

\noindent The primary aim was reached with the outcome of the article finding several qualities that are common amongst most educational games. As for the provisional objectives set, all of these were also completed albeit some not being to the standard set out in the beginning. For example, while the objective to study the impact of games was discussed in the literature study, there is still a substantial amount of avenues that needed to be considered under this topic as well - such as the psychological effects of games. Beyond that, the primary aim was reached in a satisfactory way.
\\\\
\noindent Additionally, a secondary aim of this project was to develop an artefact in the form of a computer game which required the following objectives:
\begin{enumerate}
\item Learn and understand how to use the chosen development platform and associated environments;
\item Develop a base to build other levels/scenes off of;
\begin{enumerate}
\item Development of basic scripts needed.
\end{enumerate}
\item Create a specific scene/level within the aforementioned artefact that specialises in delivering information through various audio-visual stimuli that incorporates the principles and qualities found.
\end{enumerate}

\noindent The artefact was successfully developed meaning the overall aim was indeed also met. While the first two objectives set were also met in full the last objective was only met in part. The artefact and research aspects of this project are distinct with no correlation between each other bar the main topic at hand being \textit{Entertainment Computing} and this aim sought to combine them. The research was focused on educational games and the level designed was the tutorial level which intends to teach players and as such there is a correlation. The qualities identified by the research study were implemented, such as with Reflection being done by giving the user time after each section before continuing to the next. However, the objective stated the use of both audio and visual stimuli where the final implementation of the artefact makes use of only visual. While this is a small distinction to make on a self-imposed objective, it would have been a fine inclusion to make which would have also aided users who struggle with reading.
\\\\
Another aspect of the artefact and article development to discuss is the tame management of it all. While all milestone submissions were delivered on time, the time management of the individual aspects of the project were lacking. The group aspect of this project was completed roughly three weeks before final submissions while the individual research project elements, the article and documentation, were only finalised a few days prior to the deadlines with final checks typically happing on the day of submission. As such, individual time management on this project was lacking, especially nearing deadlines while the group aspects were handled especially well - even when issues in development arose.
\\\\
Following on from that point, the group aspects of this project were exceptionally well handled. As mentioned in the sections and chapters above each group member was responsible for certain aspects of the artefact development. These were: %itemise and itemise again
\begin{table}[H]
\begin{tabular}{l|l|l}
\hline
Joshua Esterhuizen       & GC Wehmeyer          & Rickus Trollip           \\ \hline
Player Movement          & Player Animation     & Main Menu                \\
Ball physics             & Health Bar           & Enemy AI                 \\
Moving platforms         & Player Model         & Game music aquisition    \\
Portal asset development & Enemy Models         & Controller configuration \\
Teleportation            & Environment assets   & Second Level designer    \\
Pause Menu               & Third Level designer &                          \\
Level Transition         &                      &                          \\
Tutorial level designer  &                      &                          \\ \hline
\end{tabular}
\end{table}

\noindent As indicated in the lists above, there was a roughly equal contribution from each group member. There was also no major problems with group dynamics throughout the duration of this project as each member was respectful and eager to learn about the aspects which the others were responsible for. 
\\\\
The only notable incident during development was an issue with the standalone build where certain script references were not working when they did while in the editor. While this issue was a stressful issue as it concerned a cornerstone of the game, it was resolved after a few hours debugging of the issue. 

\section{Future Work Possibilities}
While the project is considered to completed as it stands, there are still aspects that can be improved or others added. In terms of the artefact, there are numerous game design techniques that could still be taken advantage of. One example of this is Culling. This is a process by which certain elements within a game are unloaded and removed while the user is not viewing or interacting with them. The Unity Editor does include a simple version of this, however, a more robust means of it could be implemented with additional research into the feature. 
\\\\
With respect to the research study and article, there are two possibilities for future work. One is to simply continue the research and expand on what was already found and including additional fields to perhaps describe new qualities. For a more practical approach, an actual serious game could be designed and developed once a suitable topic is found to present in this format.

\section{Conclusion}
The process of working through a project in this capacity was an exciting learning experience. Being able to work on a topic that has always been a personal interest was an immense help to keep the motivation of working on both an individual research assignment and a group artefact at an all time high. The process of creating a provisional plan, performing literary research and documenting the entire development process was certainly a new way of approaching a project for myself. This way of completing a project is certainly a major help to finishing the project while maintaining the original aim and scope at all times. It was certainly a valuable educational and personal experience which will be useful in all future projects I take part in. 