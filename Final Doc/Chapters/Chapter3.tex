% Chapter 3

\chapter{Development of Artefact and Article} % Main chapter title

\label{Chapter3} % For referencing the chapter elsewhere, use \ref{Chapter1} 

This chapter describes the artefact that was developed in a group effort as well as the process followed for the writing of the academic article.

\section{Description of Artefact and Article}
This research project was split into two main aspects: 
\begin{enumerate}
\item An individual research study into a chosen topic \textit{(this being the qualities needed for a computer game to be used in education)}
\item A group effort to develop a digital computer game.
\end{enumerate}

\noindent For the first aspect of this project, the resulting outcome was an academic article titled, \textit{"Linking Gamification, Ludology and Pedagogy: How to Use Serious Games for Various Knowledge Domains"}. As stipulated in the title, the fields of gamification, ludology and pedagogy were the focus of this research aspect of the project. In this article, the question posed by this project is again reiterated and sequentially answered. The article is attached to this document under Appendix  \ref{AppendixC}. 
\\\\
The artefact accompanying this research project is a digital computer came which was developed as part of a group effort as mentioned previously. While this artefact is separate from the individual research studies conducted by its' group members, it may have been influenced by the topics being researched. The artefact is titled \textit{"Puzzle Ball: Spherical Shadows"} and is further categorised as a 3D puzzle platformer with aspects of action gameplay played in first person. A playable version of the artefact is available on the website itch.io website\footnote{\url{https://josh-scg.itch.io/puzzle-ball-spherical-shadows}} and the source code of the artefact is available on GitHub\footnote{\url{https://github.com/GCWehmeyer/Spherical_Shadows}}$^{,}$\footnote{\url{https://github.com/Josh-SCG/Spherical_Shadows}}
\\\\
Before moving on the the development of the artefact, a brief look at the cycle of play will be shown. 
\\\\
Once the application is launched, the user is taken to the \textit{Main Menu} screen (depicted below) where clicking the text \textit{Play} triggers a function to start the first level - that being the the Tutorial. Each group member was responsible for the design of an individual level where the Tutorial/First level being the level undertaken in this project. The reason for this is that the research aspect of this project aligns with the goals of a tutorial level.

\begin{figure}[H]
\centering
\begin{subfigure}{0.5\textwidth}
  \centering
  \includegraphics[width=1\linewidth]{Figures/menu.png}
  \caption{The Main Menu of the Artefact}
\end{subfigure}%
\begin{subfigure}{0.5\textwidth}
  \centering
  \includegraphics[width=1\linewidth]{Figures/start.png}
  \caption{The Starting Point of the Level}
\end{subfigure}
\caption{Starting the Game}
\end{figure}

\noindent One the tutorial level has started, the user is shown the basic character movement controls on the right side of the screen and can continue forward. Text based signs such as this were used throughout the level to explain various mechanics or tasks to be completed.

\begin{figure}[H]
\centering
\centerline{\includegraphics[scale=0.33]{tutSignEg.png}}
\caption{A Further Example of the Tutorial Signs}
\end{figure}

\noindent The user must then make there way to a central area, referred to as the Hub Room, where they are given the choice in what order they would like to complete the five main tutorials. These will be discussed and shown below. As the user completes the several tutorials, a blockade that restricted movement to the next level is slowly removed as shown below.

\begin{figure}[H]
\centering
\begin{subfigure}{0.3\textwidth}
  \centering
  \includegraphics[width=1\linewidth]{Figures/barrier5.png}
  \caption{No Tutorials Completed}
\end{subfigure}%
\begin{subfigure}{0.3\textwidth}
  \centering
  \includegraphics[width=1\linewidth]{Figures/barrier3.png}
  \caption{Two Tutorials Completed}
\end{subfigure}
\begin{subfigure}{0.3\textwidth}
  \centering
  \includegraphics[width=1\linewidth]{Figures/barrier0.png}
  \caption{All Tutorials Completed}
\end{subfigure}
\caption{Progress Blocker in Tutorial}
\end{figure}

\noindent Once all the tutorials in this level are completed and the blockade removed, the user is allowed to move towards the final room where they must throw the ball object into the centre of the desk-like object with a particle effect on it.

\begin{figure}[H]
\centering
\centerline{\includegraphics[scale=0.33]{endgoal.png}}
\caption{The End Goal Object}
\end{figure}

\noindent Once the ball object collides with the end goal object, the user is taken to the next level where the main aim to to traverse the level and throw the ball object into the next end goal. This process is followed for two additional levels for three in total.

\section{Artefact Life Cycle Followed and its Different Phases}
% Briefly provide general background on the life cycle that was followed and why it is appropriate to this artefact.

\section{Description of the Development of the Artefact}
%Comprehensive report on the way in which each phase of the life cycle was applied in the development of this artefact.


%Full Level
\begin{figure}[H]
\centering
\begin{subfigure}{0.5\textwidth}
  \centering
  \includegraphics[width=1\linewidth]{Figures/fullplan.png}
  \caption{Provisional Plan}
\end{subfigure}%
\begin{subfigure}{0.5\textwidth}
  \centering
  \includegraphics[width=1\linewidth]{Figures/full.png}
  \caption{Screenshot in Unity Editor}
\end{subfigure}
\caption{Layout of Full Level (Tutorial)}
\end{figure}


%Hub
\begin{figure}[H]
\centering
\begin{subfigure}{0.5\textwidth}
  \centering
  \includegraphics[width=1\linewidth]{Figures/hubplan.png}
  \caption{Provisional Plan}
\end{subfigure}%
\begin{subfigure}{0.5\textwidth}
  \centering
  \includegraphics[width=1\linewidth]{Figures/hub.png}
  \caption{Screenshot in Artefact}
\end{subfigure}
\caption{Layout of Hub Room}
\end{figure}


%Platform
\begin{figure}[H]
\centering
\begin{subfigure}{0.5\textwidth}
  \centering
  \includegraphics[width=1\linewidth]{Figures/platformplan.png}
  \caption{Provisional Plan}
\end{subfigure}%
\begin{subfigure}{0.5\textwidth}
  \centering
  \includegraphics[width=1\linewidth]{Figures/platform.png}
  \caption{Screenshot in Artefact}
\end{subfigure}
\caption{Layout of Platforming Tutorial}
\end{figure}

%WallRun
\begin{figure}[H]
\centering
\begin{subfigure}{0.5\textwidth}
  \centering
  \includegraphics[width=1\linewidth]{Figures/wallplan.png}
  \caption{Provisional Plan}
\end{subfigure}%
\begin{subfigure}{0.5\textwidth}
  \centering
  \includegraphics[width=1\linewidth]{Figures/wall.png}
  \caption{Screenshot in Artefact}
\end{subfigure}
\caption{Layout of Wall Running Tutorial}
\end{figure}

%Throw
\begin{figure}[H]
\centering
\begin{subfigure}{0.5\textwidth}
  \centering
  \includegraphics[width=1\linewidth]{Figures/throwplan.png}
  \caption{Provisional Plan}
\end{subfigure}%
\begin{subfigure}{0.5\textwidth}
  \centering
  \includegraphics[width=1\linewidth]{Figures/throw.png}
  \caption{Screenshot in Unity Editor}
\end{subfigure}
\caption{Layout of Throw Tutorial}
\end{figure}


%Mcap
\begin{figure}[H]
\centering
\begin{subfigure}{0.5\textwidth}
  \centering
  \includegraphics[width=1\linewidth]{Figures/mcapplan.png}
  \caption{Provisional Plan}
\end{subfigure}%
\begin{subfigure}{0.5\textwidth}
  \centering
  \includegraphics[width=1\linewidth]{Figures/mcap.png}
  \caption{Screenshot in Artefact}
\end{subfigure}
\caption{Layout of Movement Tutorial}
\end{figure}

%Enemy
\begin{figure}[H]
\centering
\begin{subfigure}{0.5\textwidth}
  \centering
  \includegraphics[width=1\linewidth]{Figures/enemyplan.png}
  \caption{Provisional Plan}
\end{subfigure}%
\begin{subfigure}{0.5\textwidth}
  \centering
  \includegraphics[width=1\linewidth]{Figures/enemy.png}
  \caption{Screenshot in Unity Editor}
\end{subfigure}
\caption{Layout of Enemy Tutorial}
\end{figure}
